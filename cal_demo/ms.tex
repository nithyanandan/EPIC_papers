
% Basic setup. Most papers should leave these options alone.
\documentclass[a4paper,fleqn,usenatbib]{../mnras}

\usepackage{newtxtext,newtxmath}

% Use vector fonts, so it zooms properly in on-screen viewing software
% Don't change these lines unless you know what you are doing
\usepackage[T1]{fontenc}
\usepackage{ae,aecompl}


%%%%% PACKAGES %%%%%

% Only include extra packages if you really need them. Common packages are:
\usepackage{graphicx}	% Including figure files
\usepackage{amsmath}	% Advanced maths commands
\usepackage{amssymb}	% Extra maths symbols
\usepackage{color}

%%%%% CUSTOM COMMANDS %%%%%

\newcommand{\s}{\hat{\mathbf{s}}} % s-hat for sine-projected direction
\newcommand{\spix}{\hat{\mathbf{s}}_{\mathrm{pix}}}
\newcommand{\Cna}{\mathcal{C}^{(n)}_a}
\newcommand{\ri}{\mathbf{r}_i}
\newcommand{\ra}{\mathbf{r}_a}
\newcommand{\rb}{\mathbf{r}_b}
\newcommand{\Nant}{N_{\mathrm{ant}}}
\newcommand{\beamr}{\widetilde{W}}
\newcommand{\beamtheta}{W}
\newcommand{\Er}[1]{\widetilde{E}_{#1}}
\newcommand{\Erest}[1]{\hat{\widetilde{E}}_{#1}}
\newcommand{\V}{\widetilde{V}}

%%%%%%%%%%%%%%%%%%% TITLE PAGE %%%%%%%%%%%%%%%%%%%

% Title of the paper, and the short title which is used in the headers.
% Keep the title short and informative.
\title[E-field Parallel Imaging Calibration]{An Efficient E-field Parallel Imaging Calibration Algorithm for Next-Generation Radio Telescopes}

% The list of authors, and the short list which is used in the headers.
% If you need two or more lines of authors, add an extra line using \newauthor
\author[Beardsley et al.]{
Adam P. Beardsley,$^{1}$\thanks{E-mail: Adam.Beardsley@asu.edu}
Nithyanandan Thyagarajan,$^{1}$
Judd D. Bowman$^{1}$
\newauthor
and Miguel F. Morales$^{2}$
\\
% List of institutions
$^{1}$Arizona State University, School of Earth and Space Exploration, Tempe, AZ 85287, USA\\
$^{2}$University of Washington, Department of Physics, Seattle, WA 98195, USA\\
}

% These dates will be filled out by the publisher
\date{Accepted XXX. Received YYY; in original form ZZZ}

% Enter the current year, for the copyright statements etc.
\pubyear{2015}

% Don't change these lines
\begin{document}
\label{firstpage}
\pagerange{\pageref{firstpage}--\pageref{lastpage}}
\maketitle

% Abstract of the paper
\begin{abstract}
Abstract here (250 words)
\end{abstract}

\begin{keywords}
instrumentation: interferometers -- techniques: image processing -- techniques: interferometric
\end{keywords}


%%%%%%%%%%%%%%%%% BODY OF PAPER %%%%%%%%%%%%%%%%%%

\section{Introduction}
The field of radio astronomy is undergoing a paradigm shift toward interferometers consisting of hundreds to thousands of small, widefield antennas. This shift is necessary to match the survey speeds required for both precision cosmology experiments, and searches for fast radio transients. Many arrays with this design are already built or under construction including \cite{XXX}. 





Motivate problem of calibrating direct imagers. Reference Nithyas paper. \citep{mor11}

\section{Mathematical Framework}

Introduce notation, describe measurement equation.


\begin{equation}
\mathcal{C}^n(a,\spix,f) \equiv \left<E_a(f) E'^*(\spix,f)\right>_t
\end{equation}
$\mathcal{C}^n(a,\spix,f) \rightarrow \Cna$

\begin{equation}
\Er{a} = g_a E^T_a
\end{equation}
\begin{align}
\hat{E}(\spix) & = \frac{1}{\Nant} \sum_i e^{2\pi i \spix \cdot \ri} \Erest{i} \\
& = \frac{1}{\Nant} \sum_i e^{2\pi i \spix \cdot \ri} \sum_b \beamr_b(\ri-\rb) \Erest{b} \\
& = \frac{1}{\Nant} \sum_i e^{2\pi i \spix \cdot \ri} \sum_b \beamr_b(\ri-\rb) h^{(n)}_b g_b\Er{b}^T
\end{align}
We can then play a trick to transform the beam.
\begin{align}
\hat{E}(\spix) & = \frac{1}{\Nant} \sum_b h^{(n)}_b g_b\Er{b}^T e^{2\pi i \spix \cdot \rb}\sum_i \beamr_b(\ri-\rb)e^{2\pi i \spix \cdot (\ri-\rb)}\\
& = \frac{1}{\Nant} \sum_b h^{(n)}_b g_b\Er{b}^T e^{2\pi i \spix \cdot \rb}\beamtheta_b(\spix)
\end{align}

Plugging in, we get,
\begin{align}
\Cna & = \left<\Er{a} \hat{E}^*(\spix)\right>_t \\ 
& = \left<g_a \Er{a}^T \frac{1}{\Nant} \sum_b h^{*(n)}_b g^*_b\Er{b}^{*T} e^{-2\pi i \spix \cdot \rb} \beamtheta^*_b(\spix)\right>_t \\
\end{align}
Pulling time independent pieces out.
\begin{align}
\Cna & = \frac{g_a}{\Nant} \sum_b h^{*(n)}_b g^*_b \beamtheta^*_b(\spix) e^{-2\pi i \spix \cdot \rb} \left<\Er{a}^T \Er{b}^{*T} \right>_t \\
& = \frac{g_a}{\Nant} \sum_b h^{*(n)}_b g^*_b \beamtheta^*_b(\spix) e^{-2\pi i \spix \cdot \rb} \V^T_{ab}
\end{align}
Solve for $g^{(n+1)}_a$.
\begin{equation}
g^{(n+1)}_a = \Cna \Nant \left[ \sum_b h^{*(n)}_b g^{*(n)}_b \beamtheta^*_b(\spix) e^{-2\pi i \spix \cdot \rb} \V^T_{ab} \right]^{-1}
\end{equation}

In the case where $h_b = 1/g_b$, this simplifies slightly.
\begin{equation}
g^{(n+1)}_a = \Cna \Nant \left[ \sum_b \beamtheta^*_b(\spix) e^{-2\pi i \spix \cdot \rb} \V^T_{ab} \right]^{-1}
\end{equation}


\section{Simulation}

Describe simulation

\section{Verification}

Connect to either cramer-rao or FX solutions in some way


\section{Conclusions}
\section*{Acknowledgements}
This work has been supported by the National Science Foundation through award AST-1206552.


%%%%%%%%%%%%%%%%%%%% REFERENCES %%%%%%%%%%%%%%%%%%

% The best way to enter references is to use BibTeX:

\bibliographystyle{../mnras}
\bibliography{../epic} % if your bibtex file is called example.bib



%%%%%%%%%%%%%%%%% APPENDICES %%%%%%%%%%%%%%%%%%%%%

\appendix

\section{Some extra material}

If you want to present additional material which would interrupt the flow of the main paper,
it can be placed in an Appendix which appears after the list of references.

%%%%%%%%%%%%%%%%%%%%%%%%%%%%%%%%%%%%%%%%%%%%%%%%%%


% Don't change these lines
\bsp	% typesetting comment
\label{lastpage}
\end{document}

% End of mnras_template.tex